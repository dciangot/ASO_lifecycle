\documentclass{cmspaper}
\def\RCS$#1: #2 ${\expandafter\def\csname RCS#1\endcsname{#2}}
\RCS$Revision$
\RCS$Date$

\begin{document}
\begin{titlepage}
  \whitepaper % \internalnote{2004/XXX} % \cmsnote{2005/000} % \conferencereport{2005/000}
  \date{Revision \RCSRevision, \RCSDate}
  \title{CMS Data Handling White Paper -- Sample Paper}

  \begin{Authlist}
    Tim~Barrass, Simon~Metson\Instfoot{bristol}{University of Bristol, Bristol, UK}
    Lassi~A.~Tuura\Instfoot{neu}{Northeastern University, Boston, USA}
    % A.~Author\Iref{cern}, B.~Author\Iref{cern}, C.~Author\IAref{cern}{a},
    % D.~Author\IIref{cern}{ieph}, E.~Author\IIAref{cern}{ieph}{b},
    % F.~Author\Iref{ieph}
  \end{Authlist}

  %\Instfoot{cern}{CERN, Geneva, Switzerland}
  %\Anotfoot{a}{On leave from prison}
  %\collaboration{CMS collaboration}

  \begin{abstract}
    This white paper describes sample behaviour of the CMS data handling
    system.  It is part of the family of documents written to confuse the
    user community at large, and to conceal and obfuscate the details of
    data handling so nobody knows what is going on.
  \end{abstract} 

  %\conference{Presented at {\it Physics Rumours}, Coconut Island, April 1, 2005}
  %\submitted{Submitted to {\it Physics Rumours}}
  \note{Preliminary DRAFT version}
\end{titlepage}

\setcounter{page}{2}

\section{A Section}
Some very profoundly confusing stuff should be written here for the general
enlightment of the CMS collaboration.

\subsection{Subsection}
This is an example of subsection

\subsubsection{Subsubsection}
This is an example of subsubsection

\begin{thebibliography}{9}
  \bibitem {NOTE000} {\bf CMS Note 2005/000},
    X.Somebody et al.,
    {\em "CMS Note Template"}.
\end{thebibliography}
 
\end{document}
